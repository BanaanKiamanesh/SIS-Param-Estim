\documentclass[a4paper,10pt]{report}

% Packages
\usepackage[utf8]{inputenc}
\usepackage{amsmath, amssymb}
\usepackage{graphicx}
\usepackage{caption}
\usepackage{subcaption}
\usepackage{geometry}
\usepackage{hyperref}
\usepackage{natbib}
\geometry{margin=2.5cm}


\title{Bayesian Parameter Estimation of the Network SIS Model using MCMC Methods}
\author{ Banaan Kiamanesh, Pietro Ferraresi, Riccardo Bresolin,\\ Matteo Polo, 
Mauricio Dada Fonseca De Freitas}
\date{May 2025}

\begin{document}

\maketitle

\begin{abstract}
This project implements Bayesian parameter estimation for the Network SIS model using Markov Chain Monte Carlo (MCMC) methods. We aim to estimate key parameters such as network interaction and recovery rates, highlighting the advantages of Bayesian approaches over traditional estimation techniques.
\end{abstract}

\section{Introduction}

The spread of infectious diseases remains a critical challenge for public health systems worldwide. The SIS model captures the dynamics of diseases where immunity is temporary or non-existent. This makes the SIS framework particularly relevant for ongoing surveillance and control of endemic infections.

Our motivation to focus on the SIS model stems from its wide applicability in modeling the persistence of infectious diseases within a population, especially in interconnected environments such as social networks, transportation systems, or healthcare settings. In these contexts, network-based extensions of the SIS model allow researchers to capture the heterogeneity of contacts and transmission pathways between individuals or subpopulations. Accurate estimation of the SIS model parameters, such as transmission rates and recovery rates, is essential for developing effective intervention strategies, including vaccination, quarantine policies, or resource allocation in healthcare systems.

Lately, the relevance of this topic has been further amplified by "JOINT COMMUNICATION TO THE EUROPEAN PARLIAMENT, THE 
EUROPEAN COUNCIL, THE COUNCIL, THE EUROPEAN ECONOMIC AND 
SOCIAL COMMITTEE AND THE COMMITTEE OF THE REGIONS" a Preparedness Union Strategy to strengthen its ability to handle future pandemics and health crises. Our study aligns with these priorities by focusing on the development of Bayesian parameter estimation methods for the Network SIS model, which can enhance the reliability of epidemiological predictions under uncertainty.

In this project, we aim to implement and test a Bayesian framework for parameter estimation in the Network SIS model using Markov Chain Monte Carlo (MCMC) methods. By leveraging synthetic data and exploring the challenges of numerical identifiability, we aim to demonstrate the advantages of Bayesian approaches over classical estimation techniques and contribute to the broader effort of building resilient, data-informed public health systems.


\section{The Network SIR Model}
\subsection{Model Equations}
Present the SIS model for a network:

\[
\boxed{
\frac{dI}{dt} = \beta I (1 - I) - \gamma I
}
\]
where:
\begin{itemize}
    \item $I(t)$ is the fraction of infected individuals in the population at time $t$,
    \item $\beta$ is the transmission (infection) rate,
    \item $\gamma$ is the recovery rate.
\end{itemize}

The term $\beta I (1 - I)$ represents the rate at which susceptibles become infected, while $\gamma I$ accounts for recovery of infected individuals.



In matrix form, the system can be written compactly as:
\[
\boxed{
\frac{d\mathbf{x}}{dt} = \operatorname{diag}(1 - \mathbf{x}) \, A \, \mathbf{x} - \gamma \mathbf{x}
}
\]
where: 
\begin{itemize}
    \item $\mathbf{x} = [x_1, x_2, \dots, x_n]^T$ is the vector of infection fractions across all nodes.
    \item $A = [A_{ij}]$ is the contact rate from node $j$ to node $i$ (the adjacency matrix of the network),
    \item $\gamma$ is the recovery rate, assumed uniform across all nodes.
\end{itemize}


\subsection{Challenges in Parameter Estimation}
Discuss nonlinearity, stochasticity, and potential identifiability issues.

\section{Bayesian Framework for Parameter Estimation}
\subsection{Bayesian Formulation}
Define priors, likelihood, and posterior distribution:
\begin{align*}
    P(\theta | \text{data}) \propto P(\text{data} | \theta) P(\theta)
\end{align*}
Describe choice of priors for $\beta$, $\gamma$, etc.

\subsection{MCMC Implementation}

The network SIS model implemented in this study assumes a fixed population divided into $n = 2$ interconnected nodes, each representing a subpopulation or region. The model captures the dynamics of infection transmission using a contact matrix $A \in \mathbb{R}^{n \times n}$, where each entry $A_{ij}$ represents the transmission influence from node $j$ to node $i$. The recovery rate $\gamma$ is assumed homogeneous across all nodes. The infection fraction $x_i(t)$ at each node $i$ evolves according to the following system of differential equations:
\[
\frac{dx_i}{dt} = (1 - x_i) \sum_{j=1}^{n} A_{ij} x_j - \gamma x_i
\]
where the term $(1 - x_i) \sum_j A_{ij} x_j$ models new infections, and the term $-\gamma x_i$ accounts for recoveries. The initial conditions $y_0$ and the measurement noise variance $\sigma^2$ are assumed known, and synthetic observations are generated by adding Gaussian noise to the true infection trajectories.

The estimation of the unknown parameters, namely the contact matrix entries $A_{ij}$ and the recovery rate $\gamma$, is performed using a Bayesian inference approach with Markov Chain Monte Carlo (MCMC) sampling. A random-walk Metropolis-Hastings algorithm is employed, where candidate parameter vectors are proposed by adding Gaussian perturbations to the current state:
\[
c = x + \epsilon, \quad \epsilon \sim \mathcal{N}(0, \Sigma)
\]
with a diagonal covariance matrix $\Sigma$ controlling the proposal variance. Proposals that result in non-physical values (e.g., negative transmission or recovery rates) are immediately rejected by assigning zero prior probability. Otherwise, the candidate is accepted with probability:
\[
\alpha = \min \left(1, \frac{\text{prior}(c) \cdot \text{likelihood}(c)}{\text{prior}(x) \cdot \text{likelihood}(x)} \right)
\]
The likelihood function assumes additive Gaussian measurement noise. The posterior mean of the MCMC samples is used as the point estimate for the parameters:
\[
\hat{A} = \mathbb{E}[A], \quad \hat{\gamma} = \mathbb{E}[\gamma]
\]


\newpage
\section{Simulation and Results}

\subsection{Results}

\begin{figure}[h!]
    \centering
    % Prima immagine
    \begin{minipage}{0.65\textwidth}
        \centering
        \includegraphics[width=\textwidth]{immagini/histograms1.jpeg}
        \label{fig:immagine1}
    \end{minipage}
    \hfill
    % Seconda immagine
    \begin{minipage}{0.65\textwidth}
        \centering
        \includegraphics[width=\textwidth]{immagini/histograms2.jpeg}
        \label{fig:immagine2}
    \end{minipage}
     \hfill
    \begin{minipage}{0.65\textwidth}
        \centering
        \includegraphics[width=\textwidth]{immagini/histograms3.jpeg}
        \label{fig:immagine2}
    \end{minipage}
    \hfill
    % Seconda immagine
    \begin{minipage}{0.65\textwidth}
        \centering
        \includegraphics[width=\textwidth]{immagini/histograms4.jpeg}
        \label{fig:immagine2}
    \end{minipage}
\end{figure}


\section{Discussion}


\section{Conclusion}


\section*{Appendices}


\section*{References}
\begin{thebibliography}{9}

\bibitem[Pillonetto et al.(2003)]{pillonetto2003}
Pillonetto, G., Sparacino, G., \& Cobelli, C. (2003).
\textit{Numerical non-identifiability regions of the minimal model of glucose kinetics: superiority of Bayesian estimation}.
Mathematical Biosciences.

\bibitem[Gelman et al.(1995)]{gelman1995}
Gelman, A., Carlin, J. B., Stern, H. S., \& Rubin, D. B. (1995).
\textit{Bayesian Data Analysis}.
Chapman and Hall.

\end{thebibliography}

\end{document}
